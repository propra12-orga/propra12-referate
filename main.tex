\documentclass{programmierpraktikum}
\vorlesung{Programmierpraktikum}
\semester{Sommersemester 2012}
\betreuer{Wilfried Linder}
\subtitle{Seminarreferate}

\begin{document}
\maketitle
%
In jeder Gruppe muss je eins der folgenden Themen von einer Person be\-ar\-bei\-tet werden. Jedes Thema wird von einer zufällig ausgewählten Person aus je einer Gruppe präsentiert. Die Dauer der Präsentation sollte 15-20 Minuten betragen. Dazu muss eine Ausarbeitung bis zum Übungsgruppentermin in der Woche vom \textbf{28. Mai bis 1. Juni} angefertigt und ins Grup\-pen\-re\-po\-si\-to\-ry ge\-pusht werden. Dabei handelt es sich um eine Pflichtabgabe, die zum Bestehen Voraussetzung ist. Falsche oder plagiierte Ausarbeitungen werden nicht akzeptiert. Es wird daher empfohlen, schon vor dem Termin abzugeben. Die Präsentationen werden in den Übungsgruppen in der Woche vom \textbf{4. bis 8. Juni} vorgetragen. Der Übungsgruppenleiter legt dabei fest, wer zu welchem Termin seine Präsentation hält.
%
\section{Themen}
Folgende Themen stehen zur Auswahl:
\begin{enumerate}
  \renewcommand{\labelenumi}{\textbf{\theenumi.}}
  \item \textbf{GUI}
    \begin{itemize}
      \item Swing, SWT, AWT
      \item OpenGL, Direct3D
    \end{itemize}
  \item \textbf{Interaktion}
    \begin{itemize}
      \item Model-View-Controller
      \item Listener, Observer, Callbacks
    \end{itemize}
  \item \textbf{Netzwerkprogrammierung}
    \begin{itemize}
      \item Client/Server
      \item Sockets
    \end{itemize}
  \item \textbf{Nebenläufige Programmierung}
    \begin{itemize}
      \item Threads
      \item Synchronisation, Callbacks, Deadlocks
    \end{itemize}
  \item \textbf{Austauschbare Datenformate \& Parsing}
    \begin{itemize}
      \item JSON, XML
      \item DOM/Sax-Parser
    \end{itemize}
\end{enumerate}
\end{document}
